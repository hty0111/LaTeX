\documentclass[UTF8, a4paper]{ctexart}

%%%%%%%%%% 导入宏包 %%%%%%%%%%
\usepackage{amsmath}    % 数学公式
\usepackage{xcolor}     % 颜色
\usepackage{graphicx}   % 图片
\usepackage{geometry}   % 页边距
\usepackage{fancyhdr}   % 页眉页脚
\usepackage{setspace}   % 行间距
\usepackage{multicol}   % 多栏排版
\usepackage{lmodern}    % 字体
\usepackage{listings}	% 代码
\usepackage[hidelinks]{hyperref}   % 超链接
% \usepackage{cite}
% \usepackage{url}
% \usepackage{caption2}

%%%%%%%%%% 设置全局环境 %%%%%%%%%%
% 设置页边距
\geometry{vmargin=2.6cm, hmargin=2.45cm}
% 设置1.5倍行距
\onehalfspacing
% 设置页眉宽度
\setlength{\headheight}{13pt}
% 设置页眉页脚内容
\pagestyle{fancy}
\fancyhead[C]{胡天扬 3190105708}
\fancyfoot[C]{\thepage}

%%%%%%%%%% 自定义命令 %%%%%%%%%%
% 使文献引用以上标形式显示
\newcommand{\supercite}[1]{\textsuperscript{\cite{#1}}}
% 使section中的图、表、公式编号以A-B的形式显示
% \renewcommand{\thetable}{\arabic{section}-\arabic{table}}
% \renewcommand{\thefigure}{\arabic{section}-\arabic{figure}}
% \renewcommand{\theequation}{\arabic{section}-\arabic{equation}}
% 使图注、表注与编号之间的分隔符缺省,默认是冒号:
% \renewcommand{\captionlabeldelim}{~}

%%%%%%%%%% 内容 %%%%%%%%%%
\begin{document}

%================= 标题  ======================
\title{\huge{ 标题}}
\author{胡天扬}
\date{\today}
\maketitle  % 显示标题

%================= 目录  ======================
% \tableofcontents    % 需编译两次

%================= 摘要  ======================
\begin{flushleft}   % 左对齐
    \textbf{摘要} \\[8pt]
    \textbf{关键词}
\end{flushleft}

%================= 正文  ======================

  
\begin{multicols}{2}
\end{multicols}

\section{Section 1} \label{sec1}
% Sec 1
\subsection{Subsection 1.1}
Subsec 1.1  \footnote{第一种脚注方式}
\subsubsection{Subsubsection 1.1.1}
Subsubsec 1.1.1 \footnotemark
\footnotetext{第二种脚注方式}
\marginpar[]{边注}

% 公式
\paragraph{公式}~{}\\
行内公式:$a+b=c$\\
行间公式:
\begin{equation}
    \label{pi}
    \pi=3.14 \tag{1} 
\end{equation}
引用公式 \eqref{pi}

% 表格
\paragraph{表格}~{}
\begin{table}[!htbp]
    \caption{表格}
    \centering
    \begin{tabular}{|l|c|r|}    % 左中右对齐
        \hline
        left & center & right \\
        l & c & r\\
        \hline
    \end{tabular}
\end{table}

% 图片
\paragraph{图片}~{}
\begin{figure}[!htbp]
    \centering  % 居中
    \begin{minipage}{10em}
        \centering
        % \includegraphics[scale=1]{}
        \caption{fig1}    
    \end{minipage}
\end{figure}

% 枚举
\paragraph{枚举}
\begin{enumerate}   % 自动编号
    \item An item.
    \begin{itemize} % 不编号
        \item A nested item.
        \begin{description} % 加粗
            \item[Description] Bold. 
        \end{description}
    \end{itemize}
\end{enumerate}

% 超链接
% \href{<url>}{<text>}

% 插入代码
\paragraph{代码}
\begin{lstlisting}[language=C++]
    #include <iostream>
    int main()
    {
        std::cout << "coding" << std::endl;
    }
\end{lstlisting}

% 文献引用
\paragraph{文献引用}~{}\\
文献引用
\cite{引用}
\begin{thebibliography}{}
    \bibitem{引用} 文献引用
\end{thebibliography}

%================= 参考文献  ======================

\end{document}


